% Options for packages loaded elsewhere
\PassOptionsToPackage{unicode}{hyperref}
\PassOptionsToPackage{hyphens}{url}
%
\documentclass[
]{book}
\usepackage{amsmath,amssymb}
\usepackage{lmodern}
\usepackage{ifxetex,ifluatex}
\ifnum 0\ifxetex 1\fi\ifluatex 1\fi=0 % if pdftex
  \usepackage[T1]{fontenc}
  \usepackage[utf8]{inputenc}
  \usepackage{textcomp} % provide euro and other symbols
\else % if luatex or xetex
  \usepackage{unicode-math}
  \defaultfontfeatures{Scale=MatchLowercase}
  \defaultfontfeatures[\rmfamily]{Ligatures=TeX,Scale=1}
\fi
% Use upquote if available, for straight quotes in verbatim environments
\IfFileExists{upquote.sty}{\usepackage{upquote}}{}
\IfFileExists{microtype.sty}{% use microtype if available
  \usepackage[]{microtype}
  \UseMicrotypeSet[protrusion]{basicmath} % disable protrusion for tt fonts
}{}
\makeatletter
\@ifundefined{KOMAClassName}{% if non-KOMA class
  \IfFileExists{parskip.sty}{%
    \usepackage{parskip}
  }{% else
    \setlength{\parindent}{0pt}
    \setlength{\parskip}{6pt plus 2pt minus 1pt}}
}{% if KOMA class
  \KOMAoptions{parskip=half}}
\makeatother
\usepackage{xcolor}
\IfFileExists{xurl.sty}{\usepackage{xurl}}{} % add URL line breaks if available
\IfFileExists{bookmark.sty}{\usepackage{bookmark}}{\usepackage{hyperref}}
\hypersetup{
  pdftitle={Breaking Down Barriers: A Practitioners' Guide to Watershed Connectivity Remediation Planning},
  pdfauthor={Nick Mazany-Wright, Joshua Noseworthy, Sarah Sra, Simon M. Norris, and N. W. R Lapointe},
  hidelinks,
  pdfcreator={LaTeX via pandoc}}
\urlstyle{same} % disable monospaced font for URLs
\usepackage{longtable,booktabs,array}
\usepackage{calc} % for calculating minipage widths
% Correct order of tables after \paragraph or \subparagraph
\usepackage{etoolbox}
\makeatletter
\patchcmd\longtable{\par}{\if@noskipsec\mbox{}\fi\par}{}{}
\makeatother
% Allow footnotes in longtable head/foot
\IfFileExists{footnotehyper.sty}{\usepackage{footnotehyper}}{\usepackage{footnote}}
\makesavenoteenv{longtable}
\usepackage{graphicx}
\makeatletter
\def\maxwidth{\ifdim\Gin@nat@width>\linewidth\linewidth\else\Gin@nat@width\fi}
\def\maxheight{\ifdim\Gin@nat@height>\textheight\textheight\else\Gin@nat@height\fi}
\makeatother
% Scale images if necessary, so that they will not overflow the page
% margins by default, and it is still possible to overwrite the defaults
% using explicit options in \includegraphics[width, height, ...]{}
\setkeys{Gin}{width=\maxwidth,height=\maxheight,keepaspectratio}
% Set default figure placement to htbp
\makeatletter
\def\fps@figure{htbp}
\makeatother
\setlength{\emergencystretch}{3em} % prevent overfull lines
\providecommand{\tightlist}{%
  \setlength{\itemsep}{0pt}\setlength{\parskip}{0pt}}
\setcounter{secnumdepth}{5}
\usepackage{booktabs}
\usepackage{amsthm}
\makeatletter
\def\thm@space@setup{%
  \thm@preskip=8pt plus 2pt minus 4pt
  \thm@postskip=\thm@preskip
}
\makeatother
\ifluatex
  \usepackage{selnolig}  % disable illegal ligatures
\fi
\usepackage[]{natbib}
\bibliographystyle{apalike}

\title{Breaking Down Barriers: A Practitioners' Guide to Watershed Connectivity Remediation Planning}
\author{Nick Mazany-Wright, Joshua Noseworthy, Sarah Sra, Simon M. Norris, and N. W. R Lapointe}
\date{2021-05-17}

\begin{document}
\maketitle

{
\setcounter{tocdepth}{1}
\tableofcontents
}
\hypertarget{acknowledgements}{%
\chapter*{Acknowledgements}\label{acknowledgements}}
\addcontentsline{toc}{chapter}{Acknowledgements}

This guide document represents the culmination a planning framework conceptualized over many months of work developing Watershed Connectivity Remediation Plans in four target watersheds as part of the B.C. Fish Passage Restoration Initiative, which was funded by the BC Salmon Restoration and Innovation Fund, the Canada Nature Fund for Aquatic Species at Risk, and the RBC Bluewater Project. We were fortunate to benefit from the feedback, guidance, and wisdom of many groups and individuals throughout this process --- this publication would not have been possible without this support.
We recognize the incredible work of the BC Fish Passage Technical Working Group. The core concepts and processes presented in this guide are inspired by and build upon the strategic planning framework developed by them over the past 20 years.
We would like to thank Betty Rebellato, Craig Mount, Jason Hwang, and Eileen Jones for their invaluable counsel as part of the core project advisory team. Finally, we owe a debt of gratitude to Erik Martin, Scott Jackson, Allison Moody, Bernhard Lehner, David Cote, Steve Cooke, and Rob McLaughlin for many hours of thought-provoking discussion on freshwater connectivity theory, status assessment indicators, and barrier prioritization methods.

~

\textbf{Canadian Wildlife Federation}

\textbf{350 Michael Cowpland Drive}

\textbf{Kanata, Ontario K2M 2W1}

\textbf{Telephone: 1-877-599-5777 \textbar{} 613-599-9594}

\href{https://cwf-fcf.org/en/}{\textbf{www.cwf-fcf.org}}

\textbf{© 2021}

~

\emph{Suggested Citation:}

Mazany-Wright, N., J. Noseworthy, S. Sra, S. M. Norris, and N. W. R. Lapointe. 2021. Breaking down barriers: a practitioners' guide to watershed connectivity remediation planning. Canadian Wildlife Federation. Ottawa, Ontario, Canada. 73 pp.

\hypertarget{intro}{%
\chapter{Introduction}\label{intro}}

Placeholder

\hypertarget{concept}{%
\chapter{Conceptualizing a WCRP}\label{concept}}

There are three interconnected core concepts that WCRP coordinators must consider starting. These are, (1) the watershed for which the WCRP will be developed, (2) a preliminary list of the species that connectivity is being conserved or restored for in the watershed, and (3) a preliminary assessment of the dimensions of connectivity that the WCRP will aim to address, and the barrier types associated with those dimensions. The following sections provide an overview of each of these core concepts.

\hypertarget{watershed}{%
\section{Defining the Watershed}\label{watershed}}

The first step in conceptualizing a WCRP is to select the watershed that the plan will be developed for and to clearly define its boundary (i.e., the primary geographic scope - see Section 3.3). The choice of watershed underpins each subsequent step in the planning process, including the identification of potential stakeholders and rightsholders (see Section 2.2). There is no consistent definition of the term ``watershed''; WCRP coordinators must ultimately decide on the planning scale that best fits their desired outcomes. If starting from scratch, consider using a provincial or national watershed classification system as a starting point (e.g., \href{https://www.nrcan.gc.ca/science-data/science-research/earth-sciences/geography/topographic-information/geobase-surface-water-program-ge/watershed-boundaries/20973}{the National Hydrographic Network (NHN) Work Unit system}; see Appendix A for a list of watershed classification systems by province) and refine your selection by considering your organizational mandate, potential partner priorities, or funding eligibility requirements. Alternatively, a watershed prioritization framework can be applied to strategically identify which watersheds will most likely benefit from connectivity remediation efforts (e.g., \href{https://drive.google.com/file/d/0B52wG1Pl2gRWNWptUjZ2ME15V3M/view}{NAACC: Prioritize HUC12s for Road-Stream Crossing Surveys}, Mazany-Wright et al.~2021b). Prioritization criteria may include population status (e.g., Species at Risk), species richness, barrier density or degree of fragmentation, magnitude of cumulative anthropogenic threats, number of engaged local partners, data availability, and accessibility of the watershed, among others. Though it may be tempting to select large watersheds for efficiency in coordinating connectivity remediation efforts across a broad spatial scale (Neeson et al.~2015), developing the WCRP becomes more difficult as scale increases (i.e., increased number of partners to engage with and greater amounts of knowledge and data to manage). As such, WCRP coordinators should carefully consider the trade-offs between coordination efficiency and the realities of managing the WCRP planning process. To achieve this balance, it is recommended that watersheds not exceed the scale of a NHN work unit or a Hydrologic Unit Code 8, otherwise the planning process risks becoming less tractable.

\hypertarget{species}{%
\section{Species of Connectivity Concern}\label{species}}

Once the watershed boundary is defined, the next step is to consider the species (or in some cases, specific populations of a species) that are affected by fragmentation within the watershed. At this stage, the aim is not to finalize the list of focal species; this choice must ultimately be made by the planning team once workshops begin (i.e., ``target species'' - see Section 3.2). Identifying a preliminarily list of species will help initiate this discussion. In some cases, the species that the WCRP aims to benefit may be restricted by project funding criteria (e.g., a grant focused exclusively on aquatic Species at Risk) and will therefore need be identified before partner engagement begins. Once the preliminary list of species is identified, clearly define the life history characteristics of each species (see Table 1). If you choose to target all species in the watershed, or many species, it may make sense to group them together into target species guilds. Target species will be revisited when evaluating habitat modelling and connectivity assessment methods (see Section 4.3) and will inform the final step of conceptualizing a WCRP: defining the dimensions of connectivity and associated barrier types.

\begin{table}

\caption{(\#tab:Table 1)Typical life history characteristics of freshwater species}
\centering
\begin{tabular}[t]{ll}
\toprule{}
Life history & Description\\
\midrule{}
Diadromous & Species that migrate between the ocean and freshwater to complete their life cycles. These include species that spawn in freshwater and migrate to the ocean (anadromous) and vice versa (catadromous; Gross, Coleman, and McDowall 1988).\\
Adfluvial & Species that migrate between lakes or reservoirs and rivers (Watry and Scarnecchia 2008).\\
Fluvial & Species that migrate between mainstem rivers and tributaries (Schmetterling 2001).\\
Resident & Species that typically spend their entire life cycle near where they hatched, though may occasionally disperse (Narum et al. 2008).\\
\bottomrule{}
\end{tabular}
\end{table}

\hypertarget{methods}{%
\chapter{Methods}\label{methods}}

We describe our methods in this chapter.

\hypertarget{applications}{%
\chapter{Applications}\label{applications}}

Some \emph{significant} applications are demonstrated in this chapter.

\hypertarget{example-one}{%
\section{Example one}\label{example-one}}

\hypertarget{example-two}{%
\section{Example two}\label{example-two}}

\hypertarget{final-words}{%
\chapter{Final Words}\label{final-words}}

We have finished a nice book.

  \bibliography{book.bib,packages.bib}

\end{document}
